%%%%%%%%%%%%%%%%%%%%%%%%%%%%%%%%%%%%%%%%%
% Developer CV
% LaTeX Template
% Version 1.0 (28/1/19)
%
% This template originates from:
% http://www.LaTeXTemplates.com
%
% Authors:
% Jan Vorisek (jan@vorisek.me)
% Based on a template by Jan Küster (info@jankuester.com)
% Modified for LaTeX Templates by Vel (vel@LaTeXTemplates.com)
%
% License:
% The MIT License (see included LICENSE file)
%
%%%%%%%%%%%%%%%%%%%%%%%%%%%%%%%%%%%%%%%%%

%----------------------------------------------------------------------------------------
%	PACKAGES AND OTHER DOCUMENT CONFIGURATIONS
%----------------------------------------------------------------------------------------

\documentclass[9pt]{developercv} % Default font size, values from 8-12pt are recommended
\usepackage{multicol}
\usepackage{enumitem}

%----------------------------------------------------------------------------------------

\begin{document}

%----------------------------------------------------------------------------------------
%	TITLE AND CONTACT INFORMATION
%----------------------------------------------------------------------------------------

\begin{minipage}[t]{0.45\textwidth} % 45% of the page width for name
	\vspace{-\baselineskip} % Required for vertically aligning minipages

	% If your name is very short, use just one of the lines below
	% If your name is very long, reduce the font size or make the minipage wider and reduce the others proportionately
	\colorbox{black}{{\HUGE\textcolor{white}{\textbf{\MakeUppercase{Christopher}}}}} % First name
	
	\colorbox{black}{{\HUGE\textcolor{white}{\textbf{\MakeUppercase{Odom}}}}} % Last name
	
	\vspace{6pt}
	
	{\huge Software Engineer} % Career or current job title
\end{minipage}
\begin{minipage}[t]{0.23\textwidth} % 27.5% of the page width for the first row of icons
	\vspace{-\baselineskip} % Required for vertically aligning minipages

	% The first parameter is the FontAwesome icon name, the second is the box size and the third is the text
	% Other icons can be found by referring to fontawesome.pdf (supplied with the template) and using the word after \fa in the command for the icon you want
    \icon{MapMarker}{10}{Greater Boston, US}\vspace{-0.70em}\\
    \-\hspace{2.5em}\footnotesize{willing to relocate}\vspace{-0.16em}\\
	\icon{Phone}{10}{781 475 3804}\\
	\icon{Globe}{10}{\href{https://codom.github.io}{codom.github.io}}\\
\end{minipage}
\begin{minipage}[t]{0.28\textwidth} % 27.5% of the page width for the second row of icons
	\vspace{-\baselineskip} % Required for vertically aligning minipages
	
	% The first parameter is the FontAwesome icon name, the second is the box size and the third is the text
	% Other icons can be found by referring to fontawesome.pdf (supplied with the template) and using the word after \fa in the command for the icon you want
	\icon{At}{10}{\href{mailto:christopher.r.odom@gmail.com}{christopher.r.odom@gmail.com}}\\	
	\icon{Github}{10}{\href{https://github.com/codom}{github.com/codom}}\\
	\icon{Linkedin}{10}{\href{https://www.linkedin.com/in/christopher-r-odom/}{linkedin.com/christopher-r-odom}}\\
\end{minipage}

\vspace{0.5cm}

%----------------------------------------------------------------------------------------
%	INTRODUCTION, SKILLS AND TECHNOLOGIES
%----------------------------------------------------------------------------------------

%% \cvsect{Introduction/Skills}
%% 
%% \begin{minipage}[t]{0.4\textwidth} % 40% of the page width for the introduction text
%% 	\vspace{-\baselineskip} % Required for vertically aligning minipages
%% 
%%     I am an aspiring software developer with a wide variety of intern experience
%%     and non-professional practice. I am coming out of an experience gap, but strive
%%     to meet and exceed previous successes.
%% \end{minipage}
%% \hfill % Whitespace between
%% \begin{minipage}[t]{0.5\textwidth} % 50% of the page for the skills bar chart
%% 	\vspace{-\baselineskip} % Required for vertically aligning minipages
%% 	\begin{barchart}{5.5}
%% 		\baritem{Python}{100}
%% 		\baritem{C}{85}
%% 		\baritem{Git}{80}
%% 		\baritem{Linux}{75}
%% 	\end{barchart}
%% \end{minipage}

\cvsect{Work Experience}

\begin{entrylist}
	\entry
		{2020}
		{Software Engineer}
		{Red Hat}
		{Assisted in web API upgrades to the Ceph Dashboard\\
        \texttt{Python}\slashsep\texttt{CherryPy}}
	\entry
		{2018 -- 2019}
		{Software Engineer}
		{Draper}
		{Aided in guidance and navigation flight software development by creating
        tools for hardware and software engineers to test various software and
        hardware components to comply with strict CRS2 contract requirements (For ISS Resupply)\\
        \texttt{C}\slashsep\texttt{VxWorks}\slashsep\texttt{Python}\slashsep\texttt{Sockets}}
	\entry
		{2018 -- 2019\\\footnotesize{part time w/ classes}}
		{Lab tech}
        {ECG Lab, Umass Lowell}
		{Aided in CS education research by maintaining and upgrading systems in use at the ECG lab,
        inluding the iSense web service and it's associated appinventor plugin\\
        \texttt{Ruby on Rails}\slashsep\texttt{Linux}\slashsep\texttt{AWS}\slashsep\texttt{Java (Android)}}
\end{entrylist}

\cvsect{Github Projects}

\begin{entrylist}
	\entry
	{2023}
	{\href{https://chrisodom.org/}{Personal Website}}
    {\href{https://github.com/Codom/codom.github.io}{https://github.com/Codom/codom.github.io}}
    {
        Custom static website generator built on top of python-markdown and using three.js for modern
	    graphics.
	    \\
	    \texttt{Python}\slashsep\texttt{Js}\slashsep\texttt{GLSL}
    }

	\entry
	{2023}
	{Notes Server}
    {}
    {
        Using python, orchestrated a web server to interactively serve markdown rendered to
        html to the user while keeping track of todo items.
	    \\
	    \texttt{Python}
    }

	\entry
	{2023}
    {CLAP Plugin}
    {\href{https://github.com/Codom/SimpleGuitarAmp}{https://github.com/Codom/SimpleGuitarAmp}}
    {
        Using a mix of Zig, C, and test libraries written in Rust, created a generic guitar
        amp simulation using a combination of cubic nonlinear distortion and filter equations.
        \\
        \texttt{Zig}\slashsep\texttt{C}
    }

\end{entrylist}

\cvsect{Extra-Cirriculars}

\begin{entrylist}
	\entry
		{2019 -- 2020}
		{Fedora on Raspberry Pi}
        {}
        {
	Built a classroom-ready Fedora base install that could be used on the Raspberry Pi by students
	\begin{itemize}\itemsep=0em
		\item Bootstrapped Fedora Minimal into a useful dev environment using scripts
		\item Researched Fedora Remix image generation
	\end{itemize}
	\texttt{C}\slashsep\texttt{Shell Scripting}\slashsep\texttt{Linux Userspace}}
	\entry
		{2019 -- 2020}
		{CCDC}
        {}
        {Worked in a fast paced environment in order to propel our school's ccdc team into
	the NECCDC regional\\
        \texttt{Shell Scripting}\slashsep\texttt{Linux Defensive SecOps}}
\end{entrylist}

%----------------------------------------------------------------------------------------
%	EDUCATION
%----------------------------------------------------------------------------------------

\cvsect{Education}

\begin{entrylist}
	\entry
		{2017 -- 2021}
		{Coursework in Computer Science}
		{UMass Lowell}
		{
		Completed graduate coursework in OS, undergrad courses dealing with Architecture, ASM, Algorithms, etc.
		and completed projects related to vm and language development.
		}
\end{entrylist}

\vspace{8mm}

% \cvsect{Skills}
% 
% \vspace{-21.75mm}
% \parbox[t]{0.175\textwidth}{% 17.5% of the text width of the page
% 	\ 
% }%
% \parbox[t]{0.825\textwidth}{% 82.5% of the text width of the page
% \begin{multicols}{4}
% \begin{itemize}[leftmargin=*]
% 	\setlength\itemsep{-0.2em}
% 	\item[\bf\#] C/C++
% 	\item[\bf\#] GDB
% 	\item[\bf\#] Tracy
% 	\item[\bf\#] Web sockets
% 	\item[\bf\#] Python
% 	\item[\bf\#] Linux
% 	\item[\bf\#] CLI
% 	\item[\bf\#] Java
% 	\item[\bf\#] Bash Scripting
% 	\item[\bf\#] GLSL
% 	\item[\bf\#] three.js
% 	\item[\bf\#] Vulkan
% 	\item[\bf\#] JS
% 	\item[\bf\#] Technical communication
% 	\item[\bf\#] Fast learner
% \end{itemize}
% \end{multicols}
% }

\end{document}
